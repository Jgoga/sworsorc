\section*{Maps and Movement diagram dictionary}

\subsection*{Class: Map}
The Map class represents the army game word, and all the hex tiles it contains.
\begin{al}
	\item[Fields:] \parbox{\textwidth}{hexes - array of hexes in the world}
	\item[Methods:]	\parbox{\textwidth}{getHex(index) - get hex from numeric index}
\end{al}

\section*{HexEdge}
A HexEdge is simply an edge between two hexes. Every Hex has 6 HexEdges, while
HexEdges that aren't on the edge of the map have two Hexes it lays between.

\begin{al}
	\item[Fields:] \parbox{\textwidth}{elements - Everything currently on the hex edge}
	\item[Methods:]	\parbox{\textwidth}{edgeCrossed - called when edge is crossed, to send message to things that care about this}
\end{al}

\section*{EdgeElement}
An EdgeElement is anything that lives on a Hex Edge
\begin{enumerate}
\item[Wall] - normal wall, such as what might border a city
\item[ForceWall] - A magical wall, cast by a spell
\item[DragonWall] - A dragon created wall at a dragon tunnel
\item[ProvinceBorder] - The boundary between two provinces
\item[Gate] - A gate
\end{enumerate}

\section*{TerrainType / TerrainImprovementType}

A TerrainType is just that: the sort of Terrain that is in a Hex. Examples
include Forest, Karoo, Mountain.

A TerrainImprovementType is something that lies on top of a TerrainType,
such as a road.

\begin{al}
	\item[Methods:]	\parbox{\textwidth}{getMovementCost - Gets the movement cost modifier}
	\item[] \parbox{\textwidth}{getCombatEffect() - gets any combat modifiers from fighting on the terrain}
\end{al}


\section*{Hex}
The Hex class represents a single hex inside the game Map.

A single Hex can have several properties in the game, notably it has one or more
terrain types, 6 or less neighbor hexes, and 6 hex edges.

Additionally a Hex can be effected by magic or random events.


\begin{al}
	\item[Fields:]	\parbox{\textwidth}{neighbors - the neighboring hexes}
	\item[] \parbox{\textwidth}{edges - the neighboring edges}
	\item[] \parbox{\textwidth}{province - the province that the hex is part of}
	\item[] \parbox{\textwidth}{TerrainType - the terrain type of the hex}
	\item[] \parbox{\textwidth}{TerrainImprovementType - the terrain improvement hex}
	\item[Methods:] \parbox{\textwidth}{isCapital - returns whether the hex is a capital}
	\item[] \parbox{\textwidth}{getNumber - return the number of the hex}
	\item[] \parbox{\textwidth}{getProvince - return the province of the hex}
	\item[] \parbox{\textwidth}{getMovementCost - return the movement cost of moving a unit onto the hex}
	\item[] \parbox{\textwidth}{getCombatEffect - return the combat modifiers for fighting on the hex}
	\item[] \parbox{\textwidth}{getEdges - return the hex edges}
	\item[] \parbox{\textwidth}{getStack - return any stack occupying the hex}
	\item[] \parbox{\textwidth}{updateStack - modify the stack}
\end{al}

\section*{Movement parts of Stack}

The stack class needs a couple of methods to facilitate Movement\begin{al}
\item[Methods:] \parbox{\textwidth}{getReachableHexes - performs a floodfill and looks at the varying terrains to determine what hexes all the units in the stack can reach}
	\item[] \parbox{\textwidth}{move(index) - move to a destination hex}
\end{al}

