\chapter{TEST PLAN (TP) TEMPLATE}

{\centering\selectlanguage{english}\bfseries\color{black}
Version 1.1, April 2014
\par}

{\centering\bfseries\color{black}
TEST PLAN: SCENARIO CLASS}

{\centering\selectlanguage{english}\bfseries\color{black}
FOR
\par}


\bigskip

{\centering\selectlanguage{english}\bfseries\color{black}
Swords \& Sorcery
\par}


\bigskip


\bigskip


\bigskip

\bigskip


\bigskip


\bigskip


\bigskip

{\centering\selectlanguage{english}\bfseries\color{black}
Version 1
\par}

{\centering\selectlanguage{english}\bfseries\color{black}
6 May 2014
\par}


\bigskip


\bigskip

{\centering\selectlanguage{english}\bfseries\color{black}
Prepared for:
\par}

{\centering\selectlanguage{english}\bfseries\color{black}
Clinton Jeffery
\par}


\bigskip


\bigskip

{\centering\selectlanguage{english}\bfseries\color{black}
Prepared by:
\par}

{\centering\selectlanguage{english}\bfseries\color{black}
Tyler Jaszkowiak
\par}

{\centering\selectlanguage{english}\bfseries\color{black}
University of Idaho
\par}

{\centering\selectlanguage{english}\bfseries\color{black}
Moscow, ID \ 83844-1010
\par}

{\centering\selectlanguage{english}\bfseries\color{black}
CS383 TPD
\par}

\pagebreak

{\centering\selectlanguage{english}\bfseries\color{black}
RECORD OF CHANGES (Change History)
\par}

\begin{flushleft}
\tablehead{}
\begin{supertabular}{|m{0.5462598in}|m{0.6712598in}|m{1.4212599in}|m{0.23375985in}|m{1.7962599in}|m{0.7337598in}|m{0.6295598in}|}
\hline
~

\centering {\selectlanguage{english}\bfseries\color{black} Change}\par

\centering {\selectlanguage{english}\bfseries\color{black} Number}\par

~
 &
~

\centering \selectlanguage{english}\bfseries\color{black} Date completed
&
~

\centering {\selectlanguage{english}\bfseries\color{black} Location of
change }\par

\centering \selectlanguage{english}\bfseries\color{black} (e.g., page or
figure \#) &
~

\centering {\selectlanguage{english}\bfseries\color{black} A}\par

\centering \selectlanguage{english}\bfseries\color{black} M\newline
D  &
~

\centering {\selectlanguage{english}\bfseries\color{black} Brief
description }\par

\centering \selectlanguage{english}\bfseries\color{black} of change &
~

\centering \selectlanguage{english}\bfseries\color{black} Approved by
(initials) &
~

\centering {\bfseries\color{black} Date }\par

\centering\arraybslash\bfseries\color{black}
approved\\
~
 &
~
 &
~
 &
~
 &
~
 &
~
 &
~
\\\hline
1
 &
6 May 2014
 &
Document
 &
A
 &
Creation of Test Plan
 &
TJ
 &
6 May 2014
\\\hline
~
 &
~
 &
~
 &
~
 &
~
 &
~
 &
~
\\\hline
~
 &
~
 &
~
 &
~
 &
~
 &
~
 &
~
\\\hline
~
 &
~
 &
~
 &
~
 &
~
 &
~
 &
~
\\\hline
~
 &
~
 &
~
 &
~
 &
~
 &
~
 &
~
\\\hline
~
 &
~
 &
~
 &
~
 &
~
 &
~
 &
~
\\\hline
~
 &
~
 &
~
 &
~
 &
~
 &
~
 &
~
\\\hline
~
 &
~
 &
~
 &
~
 &
~
 &
~
 &
~
\\\hline
~
 &
~
 &
~
 &
~
 &
~
 &
~
 &
~
\\\hline
~
 &
~
 &
~
 &
~
 &
~
 &
~
 &
~
\\\hline
~
 &
~
 &
~
 &
~
 &
~
 &
~
 &
~
\\\hline
~
 &
~
 &
~
 &
~
 &
~
 &
~
 &
~
\\\hline
~
 &
~
 &
~
 &
~
 &
~
 &
~
 &
~
\\\hline
~
 &
~
 &
~
 &
~
 &
~
 &
~
 &
~
\\\hline
~
 &
~
 &
~
 &
~
 &
~
 &
~
 &
~
\\\hline


\end{supertabular}
\end{flushleft}
{\selectlanguage{english}\color{black}
A - ADDED \ M - MODIFIED \ D -- DELETED}

{\centering\selectlanguage{english}\bfseries\color{black}
[ put program /system name here ]
\par}

\pagebreak

{\centering\selectlanguage{english}\bfseries\color{black}
TABLE OF CONTENTS
\par}

{\selectlanguage{english}\bfseries\color{black}
Section\ \ Page}

\setcounter{tocdepth}{9}
\renewcommand\contentsname{}
\tableofcontents

\bigskip

\bigskip
\setcounter{page}{1}\pagestyle{Convertiv}

\section[IDENTIFIER]{\selectlanguage{english}\bfseries\color{black}
TEST PLAN IDENTIFIER}

{\selectlanguage{english}\color{black}
This test plan covers the Scenario class in the Swords \& Sorcery project for 
Dr. Jeffery's Spring 2014 CS383 class. The Scenario class is part of the Rules 
and Game Play team's work.}


\section[REFERENCES]{\selectlanguage{english}\bfseries\color{black}
REFERENCES}

{\selectlanguage{english}\color{black}
The Scenario class itself can be found at https://github.com/cjeffery/sworsorc/blob/master/src/utilities/sscharts/Scenario.java 
The automated unit tests for the class reside in https://github.com/cjeffery/sworsorc/blob/master/src/test/sschartstests/ScenReaderTest.java 
The versions refered to in this document shall be that from commit  b86744e0a0d31de79eba0f9663298e7ba1bc7196.
}



\section[INTRODUCTION]{\bfseries\color{black} INTRODUCTION}

{\selectlanguage{english}\color{black}

This test plan is to provide as much coverage as possible to the methods 
and data of the Scenario class by verifying the results of execution 
against expectations through a series of automated unit tests and a 
manual GUI test.

}

\section[TEST ITEMS]{\bfseries\color{black} TEST ITEMS}

{\selectlanguage{english}\itshape\color{black}
These are things you intend to test within the scope of this test
plan. Essentially, something you will test, a list of what is to be
tested. This can be developed from the software application
inventories as well as other sources of documentation and information.
}

{\selectlanguage{english}\color{black}
The data read by the Scenario test will be read from one of the simple 
scenario configuration files and then verified against expectations through 
the class's accessor methods. These data items include }

{\selectlanguage{english}\color{black}
\begin{itemize}
\item The scenario's name
\item Number of players
\item Game length
\item The blue sun's initial position
\item The names of armies in the scenario
\item The controlling players of these armies
\item The setup order of these armies
\item The movement order of these armies
\item Names of nations within these armies
\item Names of the neutrals in the scenario
\item The provinces controlled by a nation
\item The characters within a nation
\item The units within a player nation
\item The units within a neutral nation
\item The races of both neutrals and player nations
\item The reinforcement and replacement description strings
\item Data related to where a neutral is leaning toward
\item Whether or not a neutral accepts human sacrifice
\end{itemize}
}

{\selectlanguage{english}\color{black}
Unfortunately, the most complex functionality of the Scenario class 
cannot be tested by automated unit tests. The unit pool populator requires 
a manual check. }

\section[SOFTWARE RISK ISSUES]{\bfseries\color{black} SOFTWARE RISK ISSUES}
{\selectlanguage{english}\color{black}

Identify what software is to be tested and what the critical areas
are, such as:

\begin{itemize}
\item Depends on Java's JSON reader and the programmer's understanding of it
\item Complex data structures such as a map of maps
\item Complex loops to iterate over data structures
\item Poor documentation surrounding some of these iterators
\end{itemize}
}

\section[FEATURES TO BE TESTED]{\bfseries\color{black} FEATURES TO BE TESTED}
{\selectlanguage{english}\itshape\color{black}

This is a listing of what is to be tested from the USERS viewpoint of
what the system does. This is not a technical description of the
software, but a USERS view of the functions.

}
{\selectlanguage{english}\color{black}
From the user's perspective, much of this data reading occurs under the hood. 
In all cases except for populating the unit pool, the Scenario class does not 
manipulate any pieces of the rest of the project. Other components read the 
data from the Scenario class. Therefore, while the solar configuration relies 
on a properly-working Scenario class, this is not apparent to the user because 
solar configuration is also handled by SolarConfig and HUDInitializer classes.

Therefore, the only visible component being tested by this plan is the placement 
of units and characters into the correct provinces of the map.}

\section[FEATURES NOT TO BE TESTED]{\bfseries\color{black}
	 FEATURES NOT TO BE TESTED}
{\selectlanguage{english}\itshape\color{black}

This is a listing of what is NOT to be tested from both the Users
viewpoint of what the system does and a configuration
management/version control view. This is not a technical description
of the software, but a USERS view of the functions.

}
{\selectlanguage{english}\color{black}
All data is verified, but testing the Scenario class alone cannot ensure 
that it reaches the HUD successfully. Integration tests are required outside 
of this plan for information such as solar configuration, move order, setup order, 
game length, and diplomacy.
}

\section[APPROACH]{\bfseries\color{black} APPROACH}
{\selectlanguage{english}\color{black}
The automated unit tests rely on the JUnit tool. Code coverate metrics can 
also be acquired regarding the automated tests by running the NetBeans Java 
Code Coverage plugin on the project. This will verify all data read in.

Because units cannot be placed into the unit pool if the map is not loaded, 
automated testing of unit placement is difficult. Therefore, the placement 
of units and characters in the correct provinces will have to 
be verified manually by counting units of each type in one of the scenario's 
nations when running the game.}

\section[ITEM PASS/FAIL CRITERIA]{\bfseries\color{black}
	 ITEM PASS/FAIL CRITERIA}

{\selectlanguage{english}\color{black}
The automated tests must all pass and result in 100% instruction coverage for 
each of the class's accessor methods. One of the nations in the scenario must 
have the correct number and types of units loaded into the provinces it controls.}

\section[SUSPENSION CRITERIA]{\bfseries\color{black}
	 SUSPENSION CRITERIA AND RESUMPTION REQUIREMENTS}
{\selectlanguage{english}\color{black}
If any of the accessor method tests fails, the test must be suspended as 
this is likely due to a change in the class's interface or the scenario 
itself. Any such bug must be resolved before it makes sense to find others.}

\section[TEST DELIVERABLES]{\bfseries\color{black} TEST DELIVERABLES}

{\selectlanguage{english}\color{black}
\begin{itemize}
\item Test plan document
\item JUnit test cases
\item A report of passes/failures from JUnit
\item A coverage report from JaCoCo
\end{itemize}}

\section[REMAINING TEST TASKS]{\bfseries\color{black} REMAINING TEST TASKS}

{\selectlanguage{english}\color{black}
All other classes in the project require some level of testing as well. 
In particular, the HUD and SolarConfig related classes directly affect 
the performance of Scenario.}

\section[ENVIRONMENTAL NEEDS]{\bfseries\color{black} ENVIRONMENTAL NEEDS}

{\selectlanguage{english}\color{black}
No specific requirements exist.}

\section[STAFFING AND TRAINING NEEDS]{\bfseries\color{black}
	 STAFFING AND TRAINING NEEDS}
{\selectlanguage{english}\color{black}
Understading of NetBeans, Java, JUnit, and JaCoCo.}

\section[RESPONSIBILITIES]{\bfseries\color{black} RESPONSIBILITIES}
{\selectlanguage{english}\color{black}
Anyone may perform these tests. However, as the author of this particular 
class, it would probably be most efficent to have Tyler Jaszkowiak run 
them.}

\section[SCHEDULE]{\bfseries\color{black} SCHEDULE}
{\selectlanguage{english}\color{black}
Tests will be conducted once before the semester ends.}

\section[PLANNING RISKS AND CONTINGENCIES]{\bfseries\color{black}
	 PLANNING RISKS AND CONTINGENCIES}
{\selectlanguage{english}\color{black}
[Insert text here.]}

\section[APPROVALS]{\bfseries\color{black} APPROVALS}
{\selectlanguage{english}\color{black}
Dr. Jeffery has the final word. However, I imagine that the team 
generating the test documentation will decide which tests can be 
approved and whether to continue.}

\section[GLOSSARY]{\bfseries\color{black} GLOSSARY}
{\selectlanguage{english}\color{black}
JaCoCo: The Java Code Coverage plugin for NetBeans}

\bigskip
